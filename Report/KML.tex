
%\documentclass{beamer} %voce pode usar este modelo tambem
\documentclass[handout,t]{beamer}
\usepackage{graphicx,url}
\usepackage[brazil]{babel}   
\usepackage[utf8]{inputenc}
\batchmode
% \usepackage{pgfpages}
% \pgfpagesuselayout{4 on 1}[letterpaper,landscape,border shrink=5mm]
\usepackage{amsmath,amssymb,enumerate,epsfig,bbm,calc,color,ifthen,capt-of}
\usetheme{Berlin}
\usecolortheme{senac}

%-------------------------Titulo/Autores/Orientador------------------------------------------------
\title[Indian Institute of Technology Ropar]{Social Networks\\Course Project\\KML (Knowledge Markup Language)}
\date{}
\author[KML (Knowledge Markup Language)]{Amit Kumar Verma (mentor)\\ Paras Kumar\\ Nitin Gandhi\\}

%-------------------------Logo na parte de baixo do slide------------------------------------------

%-------------------------Este código faz o menuzinho bacana na parte superior do slide------------
\AtBeginSection[]
{
  \begin{frame}<beamer>
    \frametitle{Outline}
    \tableofcontents[currentsection]
  \end{frame}
}
\beamerdefaultoverlayspecification{<+->}
% -----------------------------------------------------------------------------
\begin{document}
% -----------------------------------------------------------------------------

%---Gerador de Sumário---------------------------------------------------------
\frame{\titlepage}
\section[]{}
\begin{frame}{Contents}
  \tableofcontents
\end{frame}
%---Fim do Sumário------------------------------------------------------------


% -----------------------------------------------------------------------------
\section{What is KML?}
\begin{frame}{What is KML?}
%introducao
KML stands for Knowledge Markup Language, a standard format for storing the data of all the Knowledge Building Portals. Knowledge Building portals like Wikipedia, Stack Exchange, GitHub, e.t.c provides their data dump in their own formats. We are trying to propose a new standard format for all these types of portals such that the analysis is easy.

\end{frame}
%------------------------------------------------------------------------------

%------------------------------------------------------------------------------
\section{Need for KML}
\begin{frame}{Need for KML}
%referencial teorico, estado da arte, etc
All these Knowledge Building portals provide their data dumps in their own format. For example, Wikipedia provides its data in an XML format with their own schema definition. Similarly, Stack Exchange provides its data dump in an XML format with different schema definition. The KML will be a new standard format for these kinds of Knowledge Building portals with a standard schema definition. The idea is to make KML flexible enough such that it can store the data of any kind of Knowledge Building portals.\end{frame}
%------------------------------------------------------------------------------

%------------------------------------------------------------------------------
\section{Components}
\begin{frame}{Components}
  \begin{itemize}
    \item KML-Compressor-Decompressor
    \item KML-for-Github
    \item Github-Spider
    \item User-Agent-Spider
    \item Wiki-Satck-KML-Downloader
  \end{itemize}
\bigskip
Source Code: \url{https://github.com/csl-622/KML}  
  
\end{frame}
%------------------------------------------------------------------------------

%------------------------------------------------------------------------------
\section{KML-Compressor-Decompressor}
\begin{frame}{KML-Compressor-Decompressor}
Compresses text section of KML and decompresses any asked revision of article.
\end{frame}
%------------------------------------------------------------------------------

%------------------------------------------------------------------------------
\section{KML-for-Github}
\begin{frame}{KML-for-Github}
Geneartes KML for github from scraped data in json format.\end{frame}
%------------------------------------------------------------------------------

%------------------------------------------------------------------------------
\section{Github-Spider}
\begin{frame}{Github-Spider}
Scrapes data from Github and stores as json files.
\end{frame}
%------------------------------------------------------------------------------

%------------------------------------------------------------------------------
\section{User-Agent-Spider}
\begin{frame}{User-Agent-Spider}
Scrapes many IP adressess to avoid blocking of scraper by Github.
\end{frame}
%------------------------------------------------------------------------------

%------------------------------------------------------------------------------
\section{Wiki-Satck-KML-Downloader}
\begin{frame}{Wiki-Satck-KML-Downloader}
Dowloads data from Wikipedia and Stackoverflow and then coverts it to KML.
\end{frame}
%------------------------------------------------------------------------------


% -----------------------------------------------------------------------------
\end{document}
%-----------------------------------------------Este comentario nunca aparecera
